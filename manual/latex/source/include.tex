\usepackage[]{graphicx,color}
\usepackage[T1]{fontenc}
\usepackage{lmodern}
\usepackage{amsmath}
\usepackage{amsfonts}
\usepackage{amssymb}
\usepackage{subfigure}
\usepackage{textpos}
\usepackage{gensymb}
\usepackage[framemethod=TikZ]{mdframed}

%code syntax highlight
\usepackage{minted}
%using round bracket
\usepackage[round]{natbib}
%using tikz plot model geomegry
\usepackage{tikz}
%using threepart table
\usepackage{booktabs}
\usepackage{threeparttable}
\usepackage[labelfont=bf]{caption} % optional
%set spacing
\usepackage{setspace}
% file/folder tree
\usepackage[edges]{forest}
\newlength\Size
%landscape table or figures
\usepackage{pdflscape} % rotate the whole page, no page number and no continuation
\usepackage{rotating} %rotate table to a landscape view
%multi row in table
\usepackage{array,multirow}


% This adds space between the Appendix number (e.g., A.101) and the title.

\renewcommand{\numberline}[1]{#1~}

% If your page numbers stick out into the righthand margin but using lengths 
% appropriate to your document. See the Introduction to the "The tocloft 
% package" for additional information.

\makeatletter
 \renewcommand{\@pnumwidth}{1.75em} 
 \renewcommand{\@tocrmarg}{3.25em}
\makeatother


% use a larger page size; otherwise, it is difficult to have complete
% code listings and output on a single page
\usepackage{fullpage}

% have an index. we use the imakeidx' replacement of the 'multind' package so
% that we can have an index of all run-time parameters separate from other
% items (if we ever wanted one)
\usepackage{imakeidx}
\makeindex[name=prmindex, title=Index of run-time parameter entries]
\makeindex[name=prmindexfull, title=Index of run-time parameters with section names]

% be able to use \note environments with a box around the text
%\usepackage{fancybox}
\newcommand{\note}[1]{
{\parindent0pt
  \begin{center}
    \shadowbox{
      \begin{minipage}[c]{0.9\linewidth}
        \textbf{Note:} #1
      \end{minipage}
    }
  \end{center}
}}


% use the listings package for code snippets. define keywords for prm files
% and for gnuplot
\usepackage{listings}
\lstset{
  language=C++,
  showstringspaces=false,
  basicstyle=\small\ttfamily,
  columns=fullflexible,
  keepspaces=true,
  frame=single,
  breaklines=true,
  postbreak=\raisebox{0ex}[0ex][0ex]{\hspace{5em}\ensuremath{\color{red}\hookrightarrow\space}}
}
\lstdefinelanguage{prmfile}{morekeywords={set,subsection,end},
                            morecomment=[l]{\#},escapeinside={\%\%}{\%},}
\lstdefinelanguage{gnuplot}{morekeywords={plot,using,title,with,set,replot},
                            morecomment=[l]{\#},}


% use the hyperref package; set the base for relative links to
% the top-level aspect directory so that we can link to
% files in the aspect tree without having to specify the
% location relative to the directory where the pdf actually
% resides
\usepackage[colorlinks,linkcolor=blue,urlcolor=blue,citecolor=blue,destlabel=true,breaklinks=true,baseurl=../]{hyperref}

% \newcommand{\HydrothermalFOAM}{{\textcolor{blue}{$\nabla$}\ HydrothermalFOAM}}
\newcommand{\HydrothermalFOAMfancy}{Hydrothermal\textcolor{blue}{$\nabla$}FOAM }
\newcommand{\foam}{HydrothermalFoam }
\newcommand{\darcyfoam}{HydrothermalSinglePhaseDarcyFoam}
\newcommand{\anhydritefoam}{HydrothermalSinglePhaseDarcyAnhydriteFoam}
% This should be pasted at the start of manuals and appropriate strings entered at locations indicated with FILL.
% Be sure the TeX file includes the following packages.
% \usepackage{graphicx}
% \usepackage{times}
% \usepackage{textpos}

\definecolor{dark_grey}{gray}{0.3}
\definecolor{FOAM_gray}{rgb}{0.94,0.94,0.94}
\definecolor{FOAM_green}{rgb}{0.27,0.49,0.36}
\def\bgcolor_Code{green!2}
\renewcommand{\familydefault}{\sfdefault}

%-------------fancy box--------------------
\newcounter{theo}[section]\setcounter{theo}{0}
\renewcommand{\thetheo}{\arabic{section}.\arabic{theo}}
\newenvironment{theo}[2][]{%
	\refstepcounter{theo}%
	\ifstrempty{#1}%
	{\mdfsetup{%
			frametitle={%
				\tikz[baseline=(current bounding box.east),outer sep=0pt]
				\node[anchor=east,rectangle,fill=green!20]
				{\strut Note~\thetheo};}}
	}%
	{\mdfsetup{%
			frametitle={%
				\tikz[baseline=(current bounding box.east),outer sep=0pt]
				\node[anchor=east,rectangle,fill=green!20]
				{\strut Note~\thetheo:~#1};}}%
	}%
	\mdfsetup{innertopmargin=10pt,linecolor=FOAM_green,%
		linewidth=2pt,topline=true,%
		frametitleaboveskip=\dimexpr-\ht\strutbox\relax
	}
	\begin{mdframed}[]\relax%
		\label{#2}}{\end{mdframed}}
%=================================

%========definition of commonly used symbols========
\def\ssd{ $^{\circ}\text{C}$ }
\def\cca{ $C_{\text{Ca}^{2+}}$ }
\def\cso4{ $C_{\text{SO}_4^{2-}}$ }
\def\ca{ $\text{Ca}^{2+}$ }
\def\so4{ $\text{SO}_4^{2-}$ }

