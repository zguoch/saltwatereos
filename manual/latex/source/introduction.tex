\section{Introduction}

\foam ---combination of Hydrothermal and \href{https://openfoam.org}{OpenFOAM} --- 
is a series of programs or toolbox tended to solve the equations that describe natural hydrothermal convection and minerial reactions in porous media, fractured-porous media.
It is primarily developed by Zhikui Guo and Lars Rüpke based on OpenFoam, which is a open source C++ library for CFD (Computational Fluid Dynamics).
\begin{figure}[htpb]
	\centering
	\documentclass[tikz,border=6pt]{standalone}

\usepackage{tikz}
% file/folder tree
\usepackage[edges]{forest}
\newlength\Size

\begin{document}


{
	\definecolor{folderbg}{RGB}{124,166,198}
	\definecolor{folderborder}{RGB}{110,144,169}
	\definecolor{leveloneC}{RGB}{255,0,0}
	\definecolor{leveltwoC}{RGB}{0,0,255}
	\definecolor{levelthreeC}{RGB}{0,0,255}
	% \newlength\Size
	\setlength\Size{4pt}
	
	\tikzset{%
		folder/.pic={%
			\filldraw [draw=folderborder, top color=folderbg!50, bottom color=folderbg] (-1.05*\Size,0.2\Size+5pt) rectangle ++(.75*\Size,-0.2\Size-5pt);
			\filldraw [draw=folderborder, top color=folderbg!50, bottom color=folderbg] (-1.15*\Size,-\Size) rectangle (1.15*\Size,\Size);
		},
		file/.pic={%
			\filldraw [draw=folderborder, top color=folderbg!5, bottom color=folderbg!10] (-\Size,.4*\Size+5pt) coordinate (a) |- (\Size,-1.2*\Size) coordinate (b) -- ++(0,1.6*\Size) coordinate (c) -- ++(-5pt,5pt) coordinate (d) -- cycle (d) |- (c) ;
		},
	}
	\forestset{%
		declare autowrapped toks={pic me}{},
		pic dir tree/.style={%
			for tree={%
				folder,
				font=\ttfamily,
				grow'=0,
			},
			before typesetting nodes={%
				for tree={%
					edge label+/.option={pic me},
				},
			},
		},
		pic me set/.code n args=2{%
			\forestset{%
				#1/.style={%
					inner xsep=2\Size,
					pic me={pic {#2}},
				}
			}
		},
		pic me set={directory}{folder},
		pic me set={file}{file},
	}
	
	\begin{forest}
		pic dir tree,
		where level=0{}{% folder icons by default; override using file for file icons
			directory,
		},
		[\textbf{HydrothermalFoam}
		[solvers \textcolor{leveloneC}{$\leftarrow$source code folder of each solver}
		[HydrothermalSinglePhaseDarcyFoam 					\textcolor{leveltwoC}{$\leftarrow$single phase} ] 
		[HydrothermalSinglePhaseDarcyAnhydriteFoam    \textcolor{leveltwoC}{$\leftarrow$anhydrite precipitation}]
		]
		[libraries \textcolor{leveloneC}{$\leftarrow$EOS, boundary conditions}
		[BoundaryConditions 		\textcolor{leveltwoC}{$\leftarrow$source code of boundary conditions}
		[HydrothermalHeatFlux 					\textcolor{cyan}{$\leftarrow \nabla T=q_{thermo}/ k_r$}]
		[SubmarinePressure 							\textcolor{cyan}{$\leftarrow \nabla p_{seafloor}=\rho_0 g z$}]
		[noFlux	 											\textcolor{cyan}{$\leftarrow \nabla p=\rho\vec{g}$}]
		[HydrothermalMassFluxPressure 		\textcolor{cyan}{$\leftarrow \nabla p=\rho \vec{g} - q_{mass}$} ]
		]
		[ThermoModels \textcolor{leveltwoC}{$\leftarrow$thermal dynamic models}
		[PureWater \textcolor{cyan}{$\leftarrow$ IAPWS-97 EOS of water}]
		]
		[freesteam-2.1 \textcolor{leveltwoC}{$\leftarrow$source code of freesteam2.1}]
		]
		[tutorials \textcolor{leveloneC}{$\leftarrow$run cases of each solvers}
		%			[HydrothermalSinglePhaseDarcyFoam
		%				[1d]
		%				[2d]
		%				[3d]
		%			]
		%			[HydrothermalSinglePhaseDarcyAnhydriteFoam
		%				[1d]
		%				[2d]
		%				[3d]
		%			]
		]
		[cookbooks \textcolor{leveloneC}{$\leftarrow$example runs of several models in literatures}
		%			[1D\_Weis2014]
		%			[2D\_Weis2014]
		]
		[manual \textcolor{leveloneC}{$\leftarrow$latex source code of manual}
		%			[manual.tex, file]
		]
		]
	\end{forest}
}

\end{document}

	\caption{Folder/file layout of \foam}
	\label{fig:filetree_main}
\end{figure}

Features of the \foam are summarized as following: 

\begin{itemize}
	\item  \textit{Original characteristics of OpenFOAM}: \foam keeps all the original characteristics of OpenFoam, for example, file structure of case, syntax of all the input files and output files, mesh, utilities, even part of varable names in the source code are kept the same. 
	This principle has two advantages, one is that it is easy to understand and to use \foam if you are a OpenFoam user. The other is that it is easy to compar and understand \foam solver and the other standard solvers in OpenFoam.
	
	\item  \textit{Mesh}: \foam supports both structured regular mesh and unsructured mesh. 
	The internal structured regular mesh tool, \mintinline{bash}{blockMesh},  is recommended for new users. While \href{http://gmsh.info}{Gmsh} is also an excellent open source unstuctured mesh generator, and there is a utility named \mintinline{bash}{gmshToFoam} can transfoam gmsh to OpenFoam mesh.
	
	\item  \textit{Boundary conditions}: even though OpenFoam has a lot of build-in boundary conditions, \textcolor{red}{we also developed some specific boundary conditions}, e.g. \mintinline{bash}{fixedHeatFlux, fixedMassFlux} 
	for the specific problem --- Hydrothermal system.
	
	\item \textit{Dimensional unit}: all units of fields and variables are follow the \href{https://en.wikipedia.org/wiki/International_System_of_Units}{international system of units}(SI), 
	which means all units are expressed by the seven base units, \textbf{kg}(mass), \textbf{m}(length), \textbf{s}(time), \textbf{K}(temperature), \textbf{mol}(quantity), \textbf{A}(current), \textbf{cd}(luminous intensity), the base units can be expressed in a row vector [kg m s K mol A cd].
	And therefore all units can be express by a row vector, e.g. unit of pressure\footnote{The SI derived unit can be found in \href{https://en.wikipedia.org/wiki/SI_derived_unit}{Wikipedia}} is $Pa=N/m^2=kg\cdot m^{-1} \cdot s^{-2}$, can be expressed by the row vector as [1 -1 -2 0 0 0 0 ], which is what we used in \foam (see cookbooks).
\end{itemize}

Based on all these features, one can easyly understand the physics and logic of a model case even the source code. The layout of folder and files are shown in figure \ref{fig:filetree_main}.


\begin{theo}[Update of \foam]{note:update}
	The \foam will be maintained and  continuous updated (fixed bugs, add new solvers, add new examples, update manual, ...) on Github repository: \href{https://github.com/zguoch/HydrothermalFoam}{https://github.com/zguoch/HydrothermalFoam}. Welcome feedback any problem on \href{https://github.com/zguoch/HydrothermalFoam/issues}{issue tracker.}
\end{theo}

\subsection{Referencing \foam}
\textcolor{red}{how to cite \foam}

\subsection{Acknowledgments}
add some grants, people, tools, ...




